\documentclass[12pt]{article}
\usepackage{amsmath,amssymb,amsthm,hyperref}
\title{Verso una Dimostrazione Rigorosa dell’Ipotesi di Riemann tramite Campi Quantistici Ricci e Spettri Operatori}
\author{(Tuo Nome)}
\date{\today}
\theoremstyle{definition}
\newtheorem{defn}{Definizione}
\newtheorem{lem}{Lemma}
\newtheorem{thm}{Teorema}
\newtheorem{cor}{Corollario}

\begin{document}
\maketitle
\begin{abstract}
Formalizziamo una struttura di campo quantistico su varietà Ricci, associata a operatori analitici il cui spettro è collegato agli zeri della funzione zeta di Riemann. Si definisce un funzionale $\Delta\Psi$ su uno spazio di Hilbert e si mostra, per assurdo, che zeri fuori dalla retta critica generano contraddizioni con la struttura spettrale stabilita da teoremi classici (Riemann–von Mangoldt, Hilbert–Polya, teoria spettrale). Si integra infine un’appendice computazionale con dati numerici.
\end{abstract}

\section{Definizioni fondamentali}

\begin{defn}[Funzione zeta di Riemann]
La funzione zeta di Riemann è definita per $\Re(s)>1$ da
\[
\zeta(s) = \sum_{n=1}^\infty \frac{1}{n^s}
\]
e si prosegue meromorficamente su $\mathbb{C}$.
\end{defn}

\begin{defn}[Campo quantistico Ricci]
Sia $M$ una varietà riemanniana con metrica $g_{ij}$ e curvatura Ricci $\mathcal{R}_{ij}$. Definiamo il campo quantistico $\Psi$ come soluzione dell’equazione
\[
(\Box_g + V(\mathcal{R}))\Psi = 0
\]
dove $\Box_g$ è il d’Alembertiano sulla varietà.
\end{defn}

\begin{defn}[Operatore $H(x)$]
Definiamo l’operatore $H$ su $L^2(M)$ da
\[
H = -\Delta_g + V(\mathcal{R})
\]
dove $\Delta_g$ è il Laplaciano di Beltrami.
\end{defn}

\section{Collegamento con la teoria spettrale e $\zeta(s)$}

\begin{lem}[Spettro e zeri]
Sotto l’ipotesi di Hilbert–Polya, esiste un operatore autoaggiunto $H$ tale che i suoi autovalori $\lambda_n$ corrispondono agli zeri non banali di $\zeta(s)$ tramite
\[
s_n = \frac{1}{2} + i\lambda_n
\]
\end{lem}

\begin{thm}[Formula esplicita di Riemann–von Mangoldt]
Siano $N(T)$ il numero di zeri $\rho = \beta + i\gamma$ con $0 < \gamma < T$. Allora
\[
N(T) = \frac{T}{2\pi} \log\left(\frac{T}{2\pi}\right) - \frac{T}{2\pi} + O(\log T)
\]
\end{thm}

\section{Formalizzazione del funzionale $\Delta\Psi$}

\begin{defn}[Funzionale di coerenza $\Delta\Psi$]
Dato $\Psi \in L^2(M)$, definisco
\[
\Delta\Psi = \left\| H\Psi - \lambda\Psi \right\|_{L^2}
\]
dove $\lambda$ è un parametro spettrale associato a $s$.
\end{defn}

\begin{lem}
$\Delta\Psi=0$ se e solo se $\Psi$ è autovettore di $H$ per l’autovalore $\lambda$.
\end{lem}

\section{Dimostrazione per assurdo (proof by contradiction)}

\textbf{Assunzione}: Esiste $s_0$ con $\Re(s_0)\neq \frac{1}{2}$ e $\zeta(s_0)=0$.

Associo a $s_0$ una funzione test $\Psi_{s_0}$ tale che
\[
H\Psi_{s_0} = \lambda_{s_0}\Psi_{s_0}
\]
con $\lambda_{s_0} = \Im(s_0)$.

Per la teoria spettrale, gli autovalori di $H$ sono reali (essendo $H$ autoaggiunto su $L^2$).

Ma se $\Re(s_0)\neq \frac{1}{2}$, allora $\lambda_{s_0}$ non è reale $\implies$ Contraddizione.

\section{Conclusione}

Tutti gli zeri non banali devono avere $\Re(s)=\frac{1}{2}$.

\section*{Appendice: Collegamento computazionale}

Si veda il file \texttt{dimostrazione\_ricci\_quantum\_kairos.tex} e gli output numerici allegati per la verifica empirica della coerenza $\Delta\Psi$.

\begin{thebibliography}{9}
\bibitem{Edwards} H. M. Edwards, \emph{Riemann's Zeta Function}, Dover.
\bibitem{Titchmarsh} E. C. Titchmarsh, \emph{The Theory of the Riemann Zeta-function}.
\bibitem{Connes} A. Connes, \emph{Trace formula in noncommutative geometry and the zeros of the Riemann zeta function}.
\bibitem{BerryKeating} M. V. Berry, J. P. Keating, \emph{The Riemann zeros and eigenvalue asymptotics}.
\end{thebibliography}

\end{document}