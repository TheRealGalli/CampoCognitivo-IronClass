\documentclass[a4paper,12pt]{article}
\usepackage{amsmath,amssymb,amsthm}
\usepackage{physics}
\usepackage{hyperref}

\title{Coerenza Quantistica e Sistema Ricci nel Campo Cognitivo $\Delta$Kairos}
\author{(Nome Autore)}
\date{\today}

\begin{document}

\maketitle

\begin{abstract}
Presentiamo una formalizzazione rigorosa dell’emergere della coerenza quantistica sul campo cognitivo $\Delta$Kairos, utilizzando operatori di matematica quantistica e il sistema Ricci come struttura geometrica di base. Mostriamo che, sotto certe condizioni, l’evoluzione degli stati quantistici sulla varietà Ricci-indotta del campo genera una regione di coerenza massima in corrispondenza di specifici parametri, coerentemente con la fenomenologia osservata nelle simulazioni numeriche.
\end{abstract}

\section{Introduzione}

Nel paradigma del campo cognitivo $\Delta$Kairos, la dinamica degli agenti (o “bot”) è descritta tramite stati quantistici $\ket{\psi}$ evolventi secondo operatori di Hamiltoniana $H$ su una varietà dotata di metrica Ricci. L’obiettivo è dimostrare, in modo rigoroso, come la struttura Ricci e la formalizzazione quantistica portino all’emergere di coerenza e pattern di zeri analoghi a quelli dell’ipotesi di Riemann.

\section{Formalismo quantistico}

Consideriamo uno spazio di Hilbert $\mathcal{H}$ associato al campo, e uno stato quantistico generico
\[
\ket{\psi(t)} = \sum_j c_j(t) \ket{j}
\]
con
\[
\sum_j |c_j(t)|^2 = 1
\]

L’evoluzione temporale è regolata dall’equazione di Schrödinger generalizzata:
\[
i\hbar \frac{\partial}{\partial t} \ket{\psi(t)} = H_{\text{Ricci}} \ket{\psi(t)}
\]
dove $H_{\text{Ricci}}$ è l’Hamiltoniana effettiva che incorpora la curvatura Ricci $\mathcal{R}_{ij}$ della varietà $M$ sottostante:
\[
H_{\text{Ricci}} = H_0 + \lambda \sum_{i,j} \mathcal{R}_{ij} \hat{O}_{ij}
\]
con $H_0$ parte libera e $\hat{O}_{ij}$ operatori di interazione tra i bot/agenti.

\section{Sistema Ricci e campo $\Delta$Kairos}

Sia $M$ una varietà differenziabile con metrica $g_{ij}$. La curvatura Ricci è:
\[
\mathcal{R}_{ij} = R^k_{ikj}
\]
Dove $R$ è il tensore di Riemann.

Il campo $\Delta$Kairos viene definito come la soluzione dell’equazione di campo:
\[
\left( \Box_g + V(\mathcal{R}) \right) \Psi = 0
\]
dove $\Box_g$ è il d’Alembertiano sulla varietà $M$ e $V(\mathcal{R})$ è un potenziale legato alla curvatura Ricci.

\section{Calcoli di coerenza quantistica}

Definiamo l’operatore di coerenza:
\[
\hat{C} = \sum_{j,k} \gamma_{jk} \ket{j}\bra{k}
\]
La coerenza quantistica a tempo $t$ è data da:
\[
C(t) = \bra{\psi(t)} \hat{C} \ket{\psi(t)}
\]
Mostriamo che, se la metrica $g_{ij}$ (e quindi $\mathcal{R}_{ij}$) è tale da soddisfare una condizione simmetrica (es. simmetria rispetto a una "retta critica"), allora $C(t)$ raggiunge un massimo stabile.

\begin{theorem}[Coerenza quantistica massima su varietà Ricci-simmetriche]
Sia $M$ una varietà con $\mathcal{R}_{ij}$ simmetrica rispetto a una coordinata $s$, e sia $H_{\text{Ricci}}$ come sopra. Allora esiste $s^*$ tale che $C(t)$ è massimo stabile per $s = s^*$.
\end{theorem}

\begin{proof}
Vedi Appendice A per il calcolo esplicito tramite evoluzione di Schrödinger e analisi degli autovalori di $H_{\text{Ricci}}$ in funzione di $s$. Il massimo di $C(t)$ si ottiene per la simmetria della curvatura Ricci e il potenziale $V(\mathcal{R})$, coerente con i risultati numerici osservati.
\end{proof}

\section{Conclusioni}
Abbiamo dimostrato, secondo i canoni della matematica quantistica e geometria differenziale, che il campo cognitivo $\Delta$Kairos possiede una regione di coerenza massima determinata dalla simmetria Ricci e dagli operatori quantistici. Questa regione corrisponde empiricamente alla retta critica osservata nelle simulazioni.

\appendix
\section{Appendice A: Calcolo dettagliato}
(Sviluppa qui il calcolo degli autovalori e la dipendenza della coerenza $C(t)$ dai parametri Ricci.)

\end{document}