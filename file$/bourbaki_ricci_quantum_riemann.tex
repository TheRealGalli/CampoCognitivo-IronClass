\documentclass[12pt]{article}
\usepackage{amsmath,amssymb,amsthm}
\title{Struttura Bourbaki: Operatore Ricci su Varietà, Funzionale di Coerenza, Zeri di $\zeta(s)$}
\author{}
\date{}

\theoremstyle{definition}
\newtheorem{defn}{Definizione}
\newtheorem{lem}{Lemma}
\newtheorem{thm}{Teorema}

\begin{document}

\section*{1. Definizioni}

\begin{defn}[Varietà Riemanniana]
Sia $M$ una varietà differenziabile di dimensione $n$, $g$ una metrica riemanniana su $M$.
\end{defn}

\begin{defn}[Tensore di Ricci]
Sia $R^i_{\phantom{i}jkl}$ il tensore di Riemann. Si pone:
\[
\operatorname{Ric}_{jk} = R^i_{\phantom{i}jik}
\]
\end{defn}

\begin{defn}[Operatore di Laplace-Beltrami]
Per $f \in C^\infty(M)$,
\[
\Delta_g f = \frac{1}{\sqrt{|g|}}\partial_i\left(\sqrt{|g|}g^{ij}\partial_j f\right)
\]
\end{defn}

\begin{defn}[Operatore Ricci quantistico]
Si definisce
\[
H = -\Delta_g + V(\operatorname{Ric})
\]
dove $V$ è funzione analitica della curvatura Ricci.
\end{defn}

\begin{defn}[Funzionale di coerenza]
Dato $\Psi \in L^2(M)$,
\[
\Delta\Psi := \|H\Psi - \lambda\Psi\|_{L^2}
\]
\end{defn}

\section*{2. Calcolo esplicito: Tensore di Ricci per la metrica sferica}

Coordinate sferiche: $x^1 = r$, $x^2 = \theta$, $x^3 = \phi$.

Metrica:
\[
g_{ij} = \begin{pmatrix}
1 & 0 & 0 \\
0 & r^2 & 0 \\
0 & 0 & r^2\sin^2\theta
\end{pmatrix}
\]

Calcolo (passaggi essenziali):

- Il tensore di Riemann su $S^2$ ha solo una componente indipendente.
- Ricci:
\[
\operatorname{Ric}_{\theta\theta} = 1
\]
\[
\operatorname{Ric}_{\phi\phi} = \sin^2\theta
\]
\[
\operatorname{Ric}_{rr} = 0
\]
In generale, su $S^2$ di raggio $r$:
\[
\operatorname{Ric}_{ij} = \frac{1}{r^2}g_{ij}
\]

\section*{3. Spettro dell'operatore $H$ su $S^2$}

\[
-\Delta_{S^2} Y_{lm} = \frac{l(l+1)}{r^2} Y_{lm}
\]
Se $V$ costante, autovalori:
\[
\lambda_l = \frac{l(l+1)}{r^2} + V
\]
\[
l = 0,1,2,\dots
\]

\section*{4. Collegamento con $\zeta(s)$ (enunciato)}

Assumendo $H$ autoaggiunto con spettro semplice, si postula che esista una corrispondenza
\[
\text{zeri di } \zeta(s) \longleftrightarrow \text{spettro di } H
\]
via $s_n = \frac{1}{2} + i\lambda_n$.

\section*{5. Dimostrazione per assurdo (outline)}

Supponiamo che $\zeta(s_0)=0$ con $\Re(s_0)\neq 1/2$.  
Per la corrispondenza spettrale, $\lambda_{s_0}$ non reale $\implies$ $H$ non autoaggiunto $\implies$ Contraddizione.

\end{document}